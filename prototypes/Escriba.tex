% -*- tex -*-

\documentclass[11pt]{article}

\usepackage[utf8]{inputenc}
\usepackage[portuges]{babel}
\usepackage{graphicx}
\usepackage[a4paper, margin=1in]{geometry}

% inclua aqui algum pacote que você precise
%\usepackage{}


\newcommand{\handout}[5]{
  \noindent
  \begin{center}
  \framebox{
    \vbox{
      \hbox to 6.2in { {\bf Teoria dos Grafos (COS242) - ECI/UFRJ } \hfill #2 }
      \vspace{8mm}
      \hbox to 6.2in { {\Large \hfill #5  \hfill} }
      \vspace{4mm}
      \hbox to 6.2in { {\em #3 \hfill #4} }
    }
  }
  \end{center}
  \vspace*{4mm}
}

\newcommand{\lecture}[4]{\handout{#1}{#2}{#3}{Escriba: #4}{Aula #1}}

\parindent 0in
\parskip 1.5ex

\begin{document}

% editar a linha abaixo com info da aula
\lecture{NUMERO --- DIA/MÊS}{2013/2}{Profs. Daniel R. Figueiredo e Ricardo Marroquim}{SEU NOME AQUI}


% sempre colocar um resumo da aula anterior e desta aula
\section{Resumo}

Na última aula nós.

Nesta aula nós.

\section{Primeira parte}

Descrever os aspectos abordados nesta aula em sequência.
Utilize as subseções para melhor dividir a aula. Seja $G=(V,E)$ um grafo 
não direcionado onde com $n=|V|$ vértices. Você deve usar o modo de equação 
para escrever funções como $O(n + m)$, ou símbolos matemáticos, como $\sum_{v \in V} d_v = 2m$.

\subsection{Primeira subseção}

Você pode colocar termos em {\bf negrito desta forma}. Você pode colocar termos em {\em itálico desta forma}.

\begin{figure}[h!]
\centering
%\includegraphics[width=0.50\textwidth]{grafo.pdf}
\caption{Exemplo de um grafo não direcionado.}\label{fig:grafo}
\end{figure}

Assim (ver acima), você pode incluir uma figura e referenciá-la no texto. Na Figura \ref{fig:grafo} temos um grafo!

Assim podemos criar uma lista de pontos:
\begin{itemize}
\item 
Este é o primeiro ponto

\item 
Este é o segundo ponto

\item 
E assim vai $\ldots$
\end{itemize}

Se você quiser enumerar sua lista, pode fazer assim:
\begin{enumerate}
\item 
Este é o primeiro ponto

\item 
Este é o segundo ponto

\item 
E assim vai $\ldots$
\end{enumerate}

\section{Próxima seção}

Aqui temos um exemplo de uma citação a algum livro, artigo, ou qualquer outra fonte de informação~\cite{KT05}.


% sempre colocar referência a livros, indicando o capítulo e página, ou mesmo outras fontes
\begin{thebibliography}{9}

\bibitem{KT05}
{\sc J.~Kleinberg} e {\sc E.~Tardos}, 
``Algorithm Design,'' 2005,
{\it Capítulo 4}, pp 219--222.

\end{thebibliography}

% **** Não apague a linha abaixo!
\end{document}
